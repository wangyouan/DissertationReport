\chapter{Speed Result}
\label{ch:speedResult}

Table~\ref{tb:runningTime} are testing result.\\

\begin{table}[h]
	\centering
	\begin{tabular}{|c|c|c|c|}
		\hline
		\textbf{Worker Numbers} & \textbf{ANN} & \textbf{LR} & \textbf{RT} \\ \hline
		\textbf{2}              & 30.356756    & 11.033707   & 5.689097    \\ \hline
		\textbf{4}              & 24.656051    & 12.359999   & 6.826605    \\ \hline
		\textbf{6}              & 28.979108    & 13.080687   & 7.436696    \\ \hline
		\textbf{8}              & 20.261038    & 23.451626   & 17.863971   \\ \hline
		\textbf{10}             & 21.283275    & 23.756444   & 17.720455   \\ \hline
		\textbf{12}             & 20.695345    & 24.761506   & 17.842046   \\ \hline
	\end{tabular}
	\caption{Running time result}
	\label{tb:runningTime}
\end{table}

From the table~\ref{tb:runningTime} we can find that, only ANN shows the trend that running time decrease as the worker number increases, the other two algorithms show an opposite trend. This may because that the computational load is not enough to show the beneficial of distributed computing. So I also did another test, use the same environment to train around 500 trees.
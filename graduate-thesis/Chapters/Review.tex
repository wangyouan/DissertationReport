\chapter{Review}
\label{ch:review}

From the very beginning of the Stock market birth, the greedy nature encourages human beings to explore how to predict its behaviors. Many theory has been developed over that.

\section{Is Stock Market Really Predictable?}
There are two opposite views on this issue \cite{2_dalton_2001}, “Random Walk” against “Efficient Market” hypothesis. Those who hold the former view believe that there are too many factors that affects stock market, which makes stock act stochastically, i.e. prediction on stock price is all but impossible. The latter theory believers think that rational investors know intrinsic values and conduct arbitrage whenever necessary, as a result, “asset prices should quickly reflect all relevant market fundamentals”\cite{1_wong_1997}.\\

The random walk hypothesis has three main different variances depending on the strength of the assumptions\cite{1_shadbolttaylor_2002}. Many tests have been developed to test this hypothesis, such as autocorrelation test\cite{lo1988stock}, Q-Statistic test\cite{box1970distribution} and variance ratio test\cite{1_shadbolttaylor_2002}. Neither of these tests stand the random walk model of asset returns.\\

EMH also can be divided into three forms\cite{1_wong_1997}, weak form, semi-strong form and strong form. Each have different assumptions and as a result, no investors can make excess profit from historical asset price or any other kind of information. Some evidences and empirical test can be found in \cite{1_keane_1983}, which support that the U.S stock market is efficient.\\

Current researches and tests on stock market strongly against the random walk hypothesis, while support EMH. It is true that “ideal” efficient is non-exist, e.g. there are always noise traders\cite{de1990noise}, who adds some kind of unpredictability to market, thus it is hard to find a perfect model that can precisely predict price changes.

\section{Traditional Ways to predict stock behavior}

Two ways have been tried to do the predict job, one is fundamental analysis, the other is technical approaches\cite{1_edwardsmagee_1997}. “Fundamental analysts believe that an investment instrument has its intrinsic value that can be derived from the behavior and performance of its company”, while “technical analysts, on the other hand, believe that the trends and patterns of an investment instrument’s price, volume, breadth, and trading activities reflect most of the relevant market information a decision maker can utilize to determine its value.”\cite{lam2004neural}, In brief, a fundamentalist utilizes business’s financial reports to predict stock behaviors, while technician just analyze historical stock trading information.\\

People also try to put this two method together. Lucas\cite{nunnostock} compares Linear regression model with SVM and in his result MAPE of both Linear regression and SVM is around 20\%. Eric\cite{alexanderstock} use Random Forest, Linear Regression and SVM, and he only make profit with the regression strategies. Nayak\cite{nayak2014impact} tested different data normalization methods’ impact on neural network, and they just use MAD as the only comparison and found that sigmoid is the most suitable function for neural network normalization. In \cite{naeini2010stock}, Mahdi test using different price as input, and find that MLP Neural Network has better performance over Linear Regression.
